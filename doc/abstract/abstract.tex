\begin{abstract}

Over the last decade, Bitcoin has seen a repeated cycle in its long term price fluctuations. These cycles typically
span over an approximate four year period where in the first year, Bitcoin experiences a cycle low in the first year
followed by a cycle peak in the fourth year. While many studies have utilized various machine learning models to
predict the price of  Bitcoin, most tend to use data related to on-chain metrics, gold price, sentiment data,
stock market data, and  Bitcoin network activity. While some of these features may be good to include in predicting
Bitcoin's price, macro economic data, such as the M2 money supply and ISM Manufacturing PMI, were not included. 
Additionally, during the investigtoin of related work, an ensemble of more than a couple of ML supervised models
was not found. This paper focuses on both including macro economic data as features into the dataset and utilizing
a Voting Regressor which ensembles AdaBoost, Random Forest, and Support Vector Machine models.

\vskip 2mm

\textbf{Keywords:} Bitcoin, Price Prediction, Classification, Machine Learning

\end{abstract}